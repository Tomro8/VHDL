% !TEX TS-program = pdflatex
% !TEX encoding = UTF-8 Unicode

% This is a simple template for a LaTeX document using the "article" class.
% See "book", "report", "letter" for other types of document.

\documentclass[]{article}

\usepackage[utf8]{inputenc} % set input encoding (not needed with XeLaTeX)

%%% PAGE DIMENSIONS
\usepackage{geometry} % to change the page dimensions
\geometry{a4paper} % or letterpaper (US) or a5paper or....
% \geometry{margin=1in} % for example, change the margins to 2 inches all round
% \geometry{landscape} % set up the page for landscape
%   read geometry.pdf for detailed page layout information

\usepackage{graphicx} % support the \includegraphics command and options

% \usepackage[parfill]{parskip} % Activate to begin paragraphs with an empty line rather than an indent

%%% PACKAGES
\usepackage{booktabs} % for much better looking tables
\usepackage{array} % for better arrays (eg matrices) in maths
\usepackage{paralist} % very flexible & customisable lists (eg. enumerate/itemize, etc.)
\usepackage{verbatim} % adds environment for commenting out blocks of text & for better verbatim
\usepackage{subfig} % make it possible to include more than one captioned figure/table in a single float
\usepackage{microtype} %makes awesome kerning and punctuation come half way out the edge of the text
\usepackage{listings} %for code listings
\usepackage{color} %for colored syntax highligting

%%% Code listing
\definecolor{mygreen}{rgb}{0,0.6,0}
\definecolor{mygray}{rgb}{0.5,0.5,0.5}
\definecolor{mymauve}{rgb}{0.58,0,0.82}
\lstset{breaklines=true,
basicstyle=\footnotesize\ttfamily,
commentstyle=\color{mygreen},
keywordstyle=\color{blue},
numberstyle=\tiny\color{mygray},
tabsize=2,
language=c
}

% to include a file as a listing: \lstinputlisting{intio.c}
% inline listing: \begin{lstlisting}[frame=single]

%%% HEADERS & FOOTERS
\usepackage{fancyhdr} % This should be set AFTER setting up the page geometry
\pagestyle{fancy} % options: empty , plain , fancy
\renewcommand{\headrulewidth}{0pt} % customise the layout...
\lhead{}\chead{}\rhead{}
\lfoot{}\cfoot{\thepage}\rfoot{}

%%% ToC (table of contents) APPEARANCE
\usepackage[nottoc,notlof,notlot]{tocbibind} % Put the bibliography in the ToC
\usepackage[titles,subfigure]{tocloft} % Alter the style of the Table of Contents
%\renewcommand{\cftsecfont}{\rmfamily\mdseries\upshape}
%\renewcommand{\cftsecpagefont}{\rmfamily\mdseries\upshape} % No bold!
\usepackage{hyperref} % use hyperlinked ToC
\hypersetup{colorlinks=true, linkcolor=black}

%%%-------------------------------------------------------------------


\title{Title Here}
\author{Oskar Weigl - ow610\\ and \\ Ryan Savitski - rs510}
%\date{} % Activate to display a given date or no date (if empty),
         % otherwise the current date is printed 

\begin{document}
\maketitle

%\renewcommand{\abstractname}{Summary}
%\begin{abstract}
%	Write the abstract here
%\end{abstract}

%\tableofcontents
%\clearpage

\section{First section}

Your text \emph{goes} here.

\subsection{A subsection}

More text.

\section{Ryans Unifinished Stuff} % (fold)
\label{sec:ryans_unifinished_stuff}

first test:
drawing lines in all 8 octants (plus horizontal/vertical)
white/black/invert lines
basic fill with black color with coordinates specified as bottom left to top right

=====
for hw this tests.....
basic operation of both blocks:
 commands handled properly 
 interfaces not violated 
 line drawing fsm and caching working
 color selection works for lines
 basic filling works (implemented in rcb)
==============================
second test:
multipoint lines (starting line segment from previous line endpoint)
black/white/invert fills

=====
for hw this tests.....
in addition to the above, tested content:
 state of cursor saved and handled properly in db
 color selection for fills
==============================
third test:
fills in all directions (i.e. the coordinates are no longer always bottom left to top right)
================
tests in hw....
 fill logic in rcb can handle all directions of start and end coordinate points
================================
fourth test:
single pixel lines/fills
single pixel wide fills
===============
tests:
 special handling of single pixel lines/fills in the db
 handling of 1-pixel wide fills in rcb
 
=================================
fifth test: 
long lines

=======
tests:
 whether draw entity has overflows in errors for long lines (note: atm doesn't check it extensively)
=======


????
fills inside one cache
fills that cross 4 caches
multiple inversions

% section ryans_unifinished_stuff (end)

\end{document}
