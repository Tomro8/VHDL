% !TEX TS-program = pdflatex
% !TEX encoding = UTF-8 Unicode

% This is a simple template for a LaTeX document using the "article" class.
% See "book", "report", "letter" for other types of document.

\documentclass[]{article}

\usepackage[utf8]{inputenc} % set input encoding (not needed with XeLaTeX)

%%% PAGE DIMENSIONS
\usepackage{geometry} % to change the page dimensions
\geometry{a4paper} % or letterpaper (US) or a5paper or....
% \geometry{margin=1in} % for example, change the margins to 2 inches all round
% \geometry{landscape} % set up the page for landscape
%   read geometry.pdf for detailed page layout information

\usepackage{graphicx} % support the \includegraphics command and options

% \usepackage[parfill]{parskip} % Activate to begin paragraphs with an empty line rather than an indent

%%% PACKAGES
\usepackage{booktabs} % for much better looking tables
\usepackage{array} % for better arrays (eg matrices) in maths
\usepackage{paralist} % very flexible & customisable lists (eg. enumerate/itemize, etc.)
\usepackage{verbatim} % adds environment for commenting out blocks of text & for better verbatim
\usepackage{subfig} % make it possible to include more than one captioned figure/table in a single float
\usepackage{microtype} %makes awesome kerning and punctuation come half way out the edge of the text
\usepackage{listings} %for code listings
\usepackage{color} %for colored syntax highligting

%%% Code listing
\definecolor{mygreen}{rgb}{0,0.6,0}
\definecolor{mygray}{rgb}{0.5,0.5,0.5}
\definecolor{mymauve}{rgb}{0.58,0,0.82}
\lstset{breaklines=true,
basicstyle=\footnotesize\ttfamily,
commentstyle=\color{mygreen},
keywordstyle=\color{blue},
numberstyle=\tiny\color{mygray},
tabsize=2,
}

% to include a file as a listing: \lstinputlisting{intio.c}
% inline listing: \begin{lstlisting}[frame=single]

%%% HEADERS & FOOTERS
\usepackage{fancyhdr} % This should be set AFTER setting up the page geometry
\pagestyle{fancy} % options: empty , plain , fancy
\renewcommand{\headrulewidth}{0pt} % customise the layout...
\lhead{}\chead{}\rhead{}
\lfoot{}\cfoot{\thepage}\rfoot{}

%%% ToC (table of contents) APPEARANCE
\usepackage[nottoc,notlof,notlot]{tocbibind} % Put the bibliography in the ToC
\usepackage[titles,subfigure]{tocloft} % Alter the style of the Table of Contents
%\renewcommand{\cftsecfont}{\rmfamily\mdseries\upshape}
%\renewcommand{\cftsecpagefont}{\rmfamily\mdseries\upshape} % No bold!
\usepackage{hyperref} % use hyperlinked ToC
\hypersetup{colorlinks=true, linkcolor=black}

%%%-------------------------------------------------------------------


\title{Complete Design of Vector Display Processor}
\author{Oskar Weigl - ow610\\ and \\ Ryan Savitski - rs510}
%\date{} % Activate to display a given date or no date (if empty),
         % otherwise the current date is printed 

\begin{document}
\maketitle

%\renewcommand{\abstractname}{Summary}
%\begin{abstract}
%	Write the abstract here
%\end{abstract}

\tableofcontents
\clearpage

\section{Operation}
The following features are implemented:
\begin{itemize}
	\item Line drawing (implemented in Draw Block):
	\begin{itemize}
		\item All colors
		\item All directions
		\item Any length (including one pixel and entire screen)
	\end{itemize}
	\item Fills (implemented in Ram Control Block)
	\begin{itemize}
		\item All colors
		\item All directions
		\item Any size
		\item Crossing any cache word boundary
	\end{itemize}
\end{itemize}

\section{Tests} % (fold)
\label{sec:tests}

The following tests always include the tests before them. I.e. test 3 includes all commands of test 2 and Test 1. This is done so we can evaluate precisely the additional code coverage rendered by each addition. The exception is the Random test, which is comprised of only random commands.

\subsection{First Test} % (fold)
\label{sub:first_test}

This initial test checks the basic operation of the hardware, with no special corner case consideration. It tests the following:
\begin{itemize}
	\item Drawing lines in all 8 octants (including horizontal/vertical).
	\item White/black/invert lines.
	\item Basic fill with black colour, with coordinates specified as bottom left to top right.
\end{itemize}

In terms of hardware, this tests:
\begin{itemize}
	\item That commands are handled properly.
	\item That interfaces are not violated.
	\item That the line drawing FSM and caching FSM are working.
	\item That colour selection works for lines.
	\item That basic filling works.
\end{itemize}

\textbf{Total coverage: 85.4\%}

% subsection first_test (end)

\subsection{Second Test} % (fold)
\label{sub:second_test}
This test checks that the hardware can handle multipoint lines. That is, line segment sequences without a move command in-between. It also extends the fill tests to handle all colors: black, white and invert.

In addition to the previous tests, this test checks the correct implementation of saving the state of cursor, and that the different colors work with the fills.

\textbf{Total coverage: 87.5\%}

% subsection second_test (end)

\subsection{Third Test} % (fold)
\label{sub:third_test}

This test checks fills in all directions. The fill implementation works by always filling from bottom left to top right, and if the incoming command has any other direction, it swaps the start and end coordinates accordingly. This test checks that the swaps are successful.

\textbf{Total coverage: 88.0\%}

% subsection third_test (end)

\subsection{Fourth Test} % (fold)
\label{sub:fourth_test}

This test checks the corner case of lines and fills that only consist of a single pixel. It also checks for fills that are a single pixel tall or a single pixel wide, but where the other dimension is of normal size.
As there is special handling of single pixel lines/fills in the draw block, this type of test will make sure that this special handling works.
This test also checks that the ram control block can handle start and end points that share the same x or y coordinate. That is, it checks that the swap comparisons do not contain any off-by-one error or similar.

\textbf{Total coverage: 89.2\%}

% subsection forth_test (end)

\subsection{Fith Test} % (fold)
\label{sub:fith_test}

This test draws lines that are longer than half the size of the screen. In particular, it draws a line from corner to corner. This checks that the variables in the \verb"draw_octant" block have enough bits to not overflow on large lines.

This test uncovered a bug in the \verb"draw_octant" code that was provided. It used a cast to \verb"signed" from \verb"unsigned" which broke if the top bit was set. That is, it used \verb"signed(xincr)" instead of \verb"signed('0' & xincr)". This bug was corrected at this point.

\textbf{Total coverage: 89.4\%}

% subsection fith_test (end)

\subsection{Sixth Test} % (fold)
\label{sub:sixth_test}

This test does a fill in the reversed direction (top right to bottom left) which is localised to a single cache block. This will check if the swapping of start and end points for the fill will occur if the fill is inside a single block.
The test also does a (reversed) 2x2 fill that perfectly straddles four cache block boundaries. This will check that fills that require only a single pixel per block works, and also checks that the correct blocks are loaded for the correct pixels.
This test also does a fill over the entire screen. This checks that the fill sequencing counters are wide enough and don't overflow.

A bug was discovered when doing a reversed fill that was localised to a single cache block. It turns out that the code didn't cover this case. The bug was fixed at this point.

\textbf{Total coverage: 92.5\%}

% subsection sixth_test (end)

\subsection{Seventh Test} % (fold)
\label{sub:seventh_test}

Using feedback from the code coverage tools of Modelsim, we discovered that some FSM transitions in the Draw Block were not exercised. That is, so far there has pretty much always been a move command between drawing of lines and fills. To make sure that these transitions are also covered, the following command pairs amend the tests:
\begin{itemize}
	\item fill $\rightarrow$ line
	\item fill $\rightarrow$ single pix
	\item line $\rightarrow$ fill
	\item line $\rightarrow$ single pix
	\item single pix $\rightarrow$ fill
	\item single pix $\rightarrow$ line
	\item move $\rightarrow$ move
\end{itemize}

\textbf{Total coverage: 92.8\%}

% subsection seventh_test (end)

\subsection{Eighth Test} % (fold)
\label{sub:eighth_test}

Further checking the code coverage, we found that the function that resolves the command to be stored in the operation cache when the new command is \verb"invert" is only applied to \verb"same".
We added some commands that would do all types of fills on top of all other types of fill, all within the same cache block. This is so that we can check that the combination of old and new commands, without flushing the cache, works.

\textbf{Total coverage: 93.0\%}

% subsection eighth_test (end)

\subsection{Random Test} % (fold)
\label{sub:random_test}

At this point, our design passes the test-bench when exposed to 10,000 randomly generated commands. We saw a small increase in code coverage from this test when compared to our synthetic tests. However, no new code paths were uncovered as the increase in coverage was only in the "expression" and "toggle" coverage metrics.

\textbf{Total coverage: 94.4\%}

% subsection random_test (end)

% section tests (end)

\section{Coverage Analysis} % (fold)
\label{sec:coverage_analysis}

\begin{figure}[htbp]
	\begin{center}
		\lstinputlisting{test_coverage/fuzz_test_report.txt}
	\end{center}
	\caption{Report of Code Coverage}
	\label{fig:report_of_coverage}
\end{figure}

Referring to the above tests, and the coverage report of the most comprehensive test, as seen in Figure~\ref{fig:report_of_coverage}, we see that not all code paths were covered.

In the Draw Block, there is a single set of statements in an \verb"if" branch that never gets executed. This is to do with the fact that the Ram Control Block never stalls the Draw Block when a move command is issued. The lack of coverage is therefore an artefact of the high efficiency of the Ram Control Block. As the statements in this branch are rather simple, a static analysis confirms that the intended operation should work if a less efficient Ram Control Block was used.

The Ram Control Block has a branch in a \verb"case" statement that is not hit. This statement corresponds to an illegal instruction that should never be issued from the Draw Block. As such, the fact that this line is not covered is in fact meritorious. As this branch, if it were to be hit, does nothing, we can deduce that if the Draw Block were to issue this illegal instruction, the fail-safe operation of going to the IDLE state should suffice.

The Pixel Word Cache has a branch in a \verb"case" statement that is not covered. This also corresponds to the same illegal instruction that is never sent. From static analysis we see that the operation that would be performed if the instruction were to be issued would be to perform no action.

The Ram Control Block's finish condition is over-conditioned. That is, some combinations of operand states in the expression that determines if the Ram Control Block asserts its \verb"finish" flag should be raised, are not covered. This simply means that, for example, it is never the case that the Ram Control Block finishes its last command with a fill that leaves the cache clean. This also applies to some other unattainable combinations.

The Ram FSM block indicates that it has uncovered statements in a function declared in a package at the top of the file. This is however not the case as putting a breakpoint on a line of this function reveals that the code does hit the statements in the function. Thus we conclude that, for the coverage report for these particular lines, Modelsim's presentation of the coverage of functions is somewhat inconsistent.

Excluding the above exceptions, which have given reasons for lack of coverage, the code has \emph{\textbf{100\% coverage}}.

\subsection{Comments on Coverage and Correctness} % (fold)
\label{sub:comments_on_coverage_and_correctness}

It is important to note that while having good code coverage is an important indicator on how well tested the hardware is, it is not the whole story. For example, Test 6 (Section~\ref{sub:sixth_test}) tests a feature that turned out to not be implemented.
As such, the bug that it uncovered would not have been uncovered using code coverage, because it can only show implemented code paths that never run, not the other way around. This is why a proper test-bench is required ans why the imagination of the tester to exercise corner cases is important.

% subsection comments_on_coverage_and_correctness (end)

% section coverage_analysis (end)

\section{Synthesis and Post Synthesis} % (fold)
\label{sec:synthesis_and_post_synthesis}

postsynth: open port problem: ciel(logstuff)

\subsection{Synthesis option space exploration} % (fold)

The design was synthesised with a variety of options under Synplify for Altera Cyclone II EP2C5 QC208 -6 speed. Requested frequency for all runs: 100 MHz. It is worth noting that auto-constrained clock was giving unexpected estimated frequencies such as retimed versions being slower.
\begin{itemize}
	\item Estimated design speed with default options: 89.8 MHz. The defaults are as follows:

\begin{itemize}
	\item device mapping enhanced optimisation
	\item default enum encoding
	\item fsm compiler
	\item resource sharing
	\item pipelining


\end{itemize}

	
\item With register retiming turned on: 93.5 Mhz. This is as expected, the critical paths were moved around by the retiming process, producing a faster design.

\item Adding fsm explorer doesn't affect the design speed as fsm state encoding is not the limiting factor of the design.

\item Automatic compile points (ACP) setting will not affect the design speed in a positive way as it is simply a multicore compilation option.

\end{itemize}

\label{sub:synthesis_option_space_exploration}

% subsection synthesis_option_space_exploration (end)

\subsection{Synthesis Warnings} % (fold)
\label{sub:synthesis_warnings}

Several synthesis warnings were encountered. Some prompted for fixes of the code, while others were assessed to be safe to ignore. The only synthesiser warnings that required fixing were inferred latches for incompletely driven signals in the combinatorial block of rcb. This was easily fixed by using a default assignment at the top of the block. The following synthesis warnings were assesed to be safe to ignore:
\begin{itemize}
	\item \verb"Variable startxblock read before being assigned." This warning is generated because \verb"startxblock" is only written in the state \verb"fill" and read in the state \verb"do_fill". This is in fact fine, as the state \verb"do_fill" can only be entered from the state \verb"fill". A way of doing this that would avoid raising this warning would be to use a signal to communicate. However, the reason for using a variable was that this communication is isolated to the process and it seems unnecessary to litter the architecture name-space.
	\item \verb"OTHERS clause is not synthesized". There are redundant others=>null defaults for case statements operating on enumerated types, this ok for future compatibility in the case of expanded enumerations.
	\item \verb"Pruning register ..." For some parts of the code, the easiest way to do in-place operations is to make a shadowing variable that always get news values at the start of the process. This means that the variables never need to save any state between clock edges, and as such they are pruned. Thus, this is the expected behaviour, and as such the warnings are safely ignored.
	\item \verb"Input xbias is unused". The actual \verb"xbias" is driven fully from within the \verb"draw_any_octant" to conform to the testbench specification without forcing unnecessary restructuring to the db architecture, the correct operation (that would not pass the testbench) will correctly drive \verb"xbias"  at the db level.
	\item \verb"Register bit ...state[5] is always 0". The synthesiser generates one-hot encoding (as verified by checking synthesis notes). The pruned bit is because we include an unreachable state called \verb"UNINITIALISED" as the first item in the state type. This is meant to serve analogously to \verb"'U'" of \verb"std_logic", that is, to detect improper reset behaviour during simulation. The fact that the synthesis tool chose to ignore this state is exactly as per design: the same way that \verb"'U'" is ignored.
\end{itemize}

% subsection synthesis_warnings (end)

\subsection{Post synthesis simulation} % (fold)
The design was simulated post-synthesis with the most complex test mentioned above and the fuzzed test. Both simulations passed the testbench normally and thus the design is considered working post-synthesis.
\label{sub:post_synthesis_simulation}

% subsection post_synthesis_simulation (end)
% section synthesis_and_post_synthesis (end)

\end{document}
